\documentclass{article}
\usepackage{amsmath, amsfonts, amssymb, amsthm, enumerate}
\theoremstyle{definition}
\newtheorem{exercise}{Exercise}[section]

\begin{document}
\addtocounter{section}{19}
\section{Metric Spaces}

\begin{exercise}
  \begin{enumerate}[(a)]
    \item In $\mathbb{R}^n$, define
    $$d'(x,y) = |x_1 - y_1| + \dots + |x_n - y_n|$$
    Show that $d'$ is a metric which induces the usual topology on $\mathbb{R}^n$.
    \item More generally, given $p\ge 1$, define
    $$d'(x,y) = \left[\sum_{i=1}^n |x_i-y_i|^p\right]^{1/p}$$
    for $x,y\in\mathbb{R}^n$. Assume that $d'$ is a metric. Show that it induces the usual topology on $\mathbb{R}^n$.
  \end{enumerate}
\end{exercise}\hrule
\begin{proof}[Solution]
  (a) We will show that $d'$ is equivalent to the square metric $\rho$. We have
  $$\rho(x,y)\le d'(x,y)\le n\rho(x,y).$$
  We have $B_{d'}(x,\varepsilon/n)\subseteq B_{d'}(x,\varepsilon)\subseteq B_\rho(x,\varepsilon)$ for all $\varepsilon > 0$, so we can apply Lemma 20.2

  (b) We have
  $$\rho(x,y)\le d'(x,y)\le n^{1/p}\rho(x,y).$$
  Now apply a similar method.
\end{proof}

\pagebreak

\begin{exercise}
  Show that $\mathbb{R}\times\mathbb{R}$ in the dictionary order is metrizable.
\end{exercise}\hrule
\begin{proof}[Solution]
  Define the metric $d$ with the equation
  $$d((a_1,a_2), (b_1,b_2)) = \begin{cases}
    \bar{d}(a_2,b_2) & a_1 = b_1 \\
    1 & a_1 \ne b_1.
  \end{cases}$$

\end{proof}

\pagebreak

\begin{exercise}
  Let $X$ be a metric space with metric $d$.
  Show that the topology induced by $d$ is the coarsest topology in which $d$ is continuous.
\end{exercise}\hrule
\begin{proof}[Solution]
  To show $d$ is continuous, we will use the $\varepsilon$-$\delta$ definition where $X\times X$ has the square metric. Choose $x,y\in X$ and $\varepsilon > 0$. Then if $x'\in B_d(x,\varepsilon/2)$ and $y'\in B_d(y,\varepsilon/2)$, we have
  \begin{align*}
    |d(x',y') - d(x,y)| &\le |d(x',y') - d(x', y)| + |d(x',y)-d(x,y)| \\
    &\le d(y,y') + d(x,x') \\
    &< \varepsilon/2 + \varepsilon/2 = \varepsilon.
  \end{align*}
  The second line is from the reverse triangle inequality.

  Now suppose $X'$ is a topological space with the same underlying set as $X$; assume $d:X'\times X'\to\mathbb{R}$ is continuous. We will show that every open ball with respect to $d$ must be open in $X'$. Fix $x,\varepsilon$ and define the function $d_x(y) = d(x,y)$, which is also continuous. Then we have
  $$d_x^{-1}((-\infty,\varepsilon)) = B_d(x,\varepsilon),$$
  thus the ball is open in $X'$, and so $X'$ is finer than $X$.
\end{proof}

\pagebreak

\begin{exercise}
  Consider the product, uniform, and box topologies on $\mathbb{R}^\omega$.
  \begin{enumerate}[(a)]
    \item In which topologies are the following functions $\mathbb{R}\to\mathbb{R}^\omega$ continuous?
    \begin{itemize}
      \item[] $f(t) = (t,2t,3t,\dots),$
      \item[] $g(t) = (t,t,t,\dots),$
      \item[] $h(t) = (t,\frac{1}{2}t,\frac{1}{3}t,\dots).$
    \end{itemize}
    \item In which topologies do the following sequences converge?
    \begin{alignat*}{3}
      w_1 &= (1,1,1,1,\dots),&\qquad x_1 &= (1,1,1,1,\dots) \\
      w_2 &= (0,2,2,2,\dots),&\qquad x_2 &= (0,\textstyle\frac{1}{2},\frac{1}{2},\frac{1}{2},\dots) \\
      w_3 &= (0,0,3,3,\dots),&\qquad x_3 &= (0,0,\textstyle\frac{1}{3},\frac{1}{3},\dots) \\
      & \dots && \dots \\
      y_1 &= (1,0,0,0,\dots),&\qquad z_1 &= (1,1,0,0,\dots) \\
      y_2 &= (\textstyle\frac{1}{2},\frac{1}{2},0,0,\dots),&\qquad z_2 &= (\textstyle\frac{1}{2},\frac{1}{2},0,0,\dots) \\
      y_3 &= (\textstyle\frac{1}{3},\frac{1}{3},\frac{1}{3},0,\dots),&\qquad z_3 &= (\textstyle\frac{1}{3},\frac{1}{3},0,0,\dots).
    \end{alignat*}
  \end{enumerate}
\end{exercise}\hrule
\begin{proof}[Solution]
  Recall that the product topology is finer than the uniform topology, which is finer than the box topology.

  (a) None of the three functions are continuous in the box topology, which can be seen by noting that the inverse images of $\prod (-1/n^2,1/n^2)$ are all equal to $\{0\}$. Also, all three functions are continuous in the product topology because their components are continuous.

  Both $g$ and $h$ are continuous in the uniform topology. To see this, let $x\in\mathbb{R}$ and $\varepsilon>0$. Then if $|x - a| < \varepsilon$, then
  \begin{align*}
    \bar{\rho}(g(x), g(a)) &= \sup \min\{|x-a|,1\} \le |x-a| < \varepsilon \\
    \bar{\rho}(h(x), h(a)) &= \sup_{n\in\mathbb{N}}\min\left\{\frac{1}{n}|x-a|,1\right\} \le |x-a| < \varepsilon.
  \end{align*}

  On the other hand, $f$ is not continuous in the uniform topology, for example at $0$. For any $\delta > 0$, we have
  $$\bar{\rho}(f(0),f(\delta/2)) = \sup_{n\in\mathbb{N}}\min\{\delta n/2, 1\} = 1.$$
  The idea is that for any small change in the input, the output still changes by a large amount at large enough indexes.

  (b) In the product topology, each sequence converges to the constant zero sequence, which is seen by looking at the indices individually. This is useful because now the sequences can only converge to $(0,0,\dots)$ in the other two finer topologies.

  In the box topology, the open set $\prod_{n\in\mathbb{N}} (-1/n,1/n)$ contains no terms in any of $w,x,y$, so none of those converge in the box topology.
  
  We can see that $z$ converges in the box topology. Let $U = \prod_{n\in\mathbb{N}}U_n$ be a product of open intervals containing $(0,0,\dots)$. Then there exists some $N\in\mathbb{N}$ such that $\frac{1}{N}\in U_1\cap U_2$, and we have $z_n\in U$ for all $n\ge N$. The sequence $(z_n)$ therefore converges in the uniform topology as well.

  Now we show that $(x_n)$ and $(y_n)$ converge in the uniform topology. Let $\varepsilon > 0$. Then letting $N = \frac{1}{\varepsilon}$, for all $n > N$ we have
  $$\bar{\rho}(\vec{0},x_{n}) = \frac{1}{n} < \varepsilon,$$
  and similarly with $y_n$. On the other hand, we have $\bar{\rho}(\vec{0},w_n) = n$, which means the distance between $w_n$ and $(0,0,\dots)$ actually increases in the uniform metric.
\end{proof}

\pagebreak

\begin{exercise}
  Let $\mathbb{R}^\infty$ be the subset of $\mathbb{R}^\omega$ consisting of all sequences that are eventually zero. What is the closure of $\mathbb{R}^\infty$ in $\mathbb{R}^\omega$ in the uniform topology?
\end{exercise}\hrule
\begin{proof}[Solution]
  For a sequence $(a_n)$ to be in the closure, we must have $B_{\bar{\rho}}((a_n), \varepsilon)$ intersecting some almost-zero sequence for all $\varepsilon > 0$. This is equivalent to having $|a_n| < \varepsilon$ for almost all $n$, for each $\varepsilon > 0$, which is the same as saying $(a_n)\to 0$. Therefore, the closure of $\mathbb{R}^{\infty}$ is the collection of sequences converging to zero.
\end{proof}

\pagebreak

\begin{exercise}
  Let $\bar{\rho}$ be the uniform metric on $\mathbb{R}^\omega$. Given $x = (x_1,x_2,\dots)\in \mathbb{R}^\omega$ and given $0 < \varepsilon < 1$, let
  $$U(x,\varepsilon) = \prod_{n=1}^\infty (x_n-\varepsilon, x_n+\varepsilon).$$
  \begin{enumerate}[(a)]
    \item Show that $U(x,\varepsilon)$ is not the $\varepsilon$-ball $B_{\bar{\rho}}(x,\varepsilon)$.
    \item Show that $U(x,\varepsilon)$ is not even open in the uniform metric.
    \item Show that
    $$B_{\bar{\rho}}(x,\varepsilon) = \bigcup_{\delta < \varepsilon} U(x,\delta).$$
  \end{enumerate}
\end{exercise}\hrule
\begin{proof}[Solution]
  (b) Let $a_n = x_n + \varepsilon(1-1/n)$. Then every neighborhood of $a_n$ contains a copy of $a_n$ shifted in the positive direction, which is not in $U(x,\varepsilon)$. This also proves part (a).

  (c) Let $(a_n)\in B_{\bar{\rho}}(x,\varepsilon)$. Then we have
  $$\sup_{n\in\mathbb{N}} |a_n - x_n| < \varepsilon,$$
  which means there is some $\delta$ such that for all $n$ we have $|a_n - x_n| < \delta < \varepsilon$. In other words, 
  $$(a_n) \in B_{\bar{\rho}}(x,\delta)\subseteq \bigcup_{\delta < \varepsilon} B_{\bar{\rho}}(x,\delta).$$
  The inclusion in the other direction is even easier to see.
\end{proof}

\pagebreak

\begin{exercise}
  Let $h:\mathbb{R}^\omega\to\mathbb{R}^\omega$ be defined by
  $$h(x_1,x_2,\dots) = (a_1x_1 + b_1,a_2x_2 + b_2,\dots),$$
  where $\mathbb{R}^\omega$ is given the uniform topology.
  Under what conditions on the numbers $a_i$ and $b_i$ is $h$ continuous? A homeomorphism?
\end{exercise}\hrule
\begin{proof}[Solution]
  $h$ is continuous when the $a_n$ are bounded, and $h$ is a homeomorphism when the $a_n$ are additionally bounded away from zero.
\end{proof}

\pagebreak

\begin{exercise}
  Let $X$ be the subset of $\mathbb{R}^\omega$ consisting of all sequences $x$ such that $\sum x_i^2$ converges. Then the formula
  $$d(x,y) = \left[\sum_{i=1}^{\infty} (x_i - y_i)^2\right]^{1/2}$$
  defines a metric on $X$. (See Exercise 10.) On $X$ we have the three topologies it inherits from the box, uniform, and product topologies on $\mathbb{R}^\omega$. We also have the topology given by the metric $d$, which we call the $\ell^2$-topology.
  \begin{enumerate}[(a)]
    \item Show that on $X$, we have the inclusions
    $$\text{box topology} \supset \ell^2\text{-topology}\supset\text{uniform topology}.$$
    \item The set $\mathbb{R}^\infty$ of all sequences that are eventually zero is contained in $X$. Show that the four topologies inherited by $\mathbb{R}^{\infty}$ from $X$ are distinct.
    \item The set
    $$H = \prod_{n=1}^{\infty}[0,1/n]$$
    is contained in $X$, it is sometimes called the Hilbert cube. Compare the four topologies that $H$ inherits from $X$.
  \end{enumerate}
\end{exercise}\hrule
\begin{proof}[Solution]
  (a) Let $B_d(x,\epsilon)$ be an open ball in the $\ell^2$-topology. Define
  $$U = \prod U_n = \prod_{n=1}^{\infty}(x_n - \epsilon/2^n, x_n + \epsilon/2^n).$$
  Then if $a\in U$, we have
  \begin{align*}
    d(a,x)^2 &= \sum_{n=1}^{\infty}(a_n-x_n)^2 \\
    &< \sum_{n=1}^{\infty}\frac{\epsilon^2}{4^n} \\
    &= \epsilon^2 / 3 < \epsilon^2 \\
    d(a,x) &\le \epsilon,
  \end{align*}
  so that $a\in B_d(x,\epsilon)$. Therefore, $U\subseteq B_d(x,\epsilon)$, showing that the box topology is finer than the $\ell^2$-topology.

  The second inclusion follows from $d(x,y)\ge \bar{\rho}(x,y)$.

  (b) The set $\mathbb{R}^\infty\cap\prod_{n=1}^{\infty} (-1/n,1/n)$ is open in the box topology, but there is no neighborhood of $(0,0,\dots)$ contained in it. This is because for any $B_d(\vec{0},\varepsilon)$, the sequences which are all zero except for one $\epsilon/2$ term are in this neighborhood, but not all are in $U$.

  The set $\mathbb{R}^\infty\cap B_d(\vec{0}, 1/2)$ is clearly open in the $\ell^2$-topology, but for any $\epsilon > 0$ we can create a sequence $(a_n)$ with enough $\epsilon/2$ terms so that it's in $B_{\bar{\rho}}(\vec{0},\epsilon)$, but not $B_d(\vec{0}, 1/2)$.

  Finally, open balls in the uniform topology are certainly not open in the product topology.

  (c) The set $\prod_{n=1}^\infty [0,\frac{1}{2n})$ is open in the inherited box topology, but I claim it's not open in the $\ell^2$-topology. This is because the sequence
  $$(\underbrace{0,0,\dots}_{\text{$n-1$}},1/n,0,0,\dots)$$
  is not in the product above, but can be made arbitrarily close to 0 in the $\ell^2$-topology's metric.

  Fix $\epsilon>0$; let $U = H\cap B_{d}(x,\epsilon)$ be a neighborhood of $x$ in the $\ell^2$-topology. Choose a positive integer $N$ such that $2/N < \epsilon^2$. Then we have
  \begin{align*}
    a\in H\cap B_{\bar{\rho}}(x,1/N) &\implies \sup |x_n - a_n| < 1/N \\
    \implies d(x,a)^2 = \sum_{n=1}^{\infty}(x_n - a_n)^2 &< N(1/N^2) + \sum_{n=N+1}^{\infty}\frac{1}{n^2} \\
    &< 1/N + \sum_{n=N}^{\infty}\frac{1}{n(n+1)} \\
    &= 2/N < \epsilon^2,
  \end{align*}
  so that $a\in B_d(x,\epsilon)$. Therefore, the $\ell^2$-topology and the uniform topology are the same in the Hilbert cube.

  Finally, fix $\epsilon > 0$; let $U = H\cap B_{\bar{\rho}}(x,2\epsilon)$ be a neighborhood of $x$ in the uniform topology. In particular, $H\cap U(x,\epsilon)$ is contained in $U$, and all but finitely many of it's product components reduce to the intervals of $H$. Therefore, we can change these components of $U(x,\epsilon)$ to all of $\mathbb{R}$ without affecting the intersection with $H$, thus this set is open in the product topology. This shows that the uniform and product topology are the same on the Hilbert cube.
\end{proof}

\pagebreak

\begin{exercise}
  Show that the euclidean metric on $\mathbb{R}^n$ is a metric.
\end{exercise}
\hrule
\begin{proof}[Solution]
  We will prove a variant of the triangle inequality. The properties used are from any course in linear algebra.
  \begin{align*}
    \Vert x + y \Vert^2 &= (x+y)\cdot(x+y) \\
    &= x\cdot x + 2(x\cdot y) + y\cdot y \\
    &\le \Vert x\Vert^2 + 2\Vert x\Vert\Vert y\Vert + \Vert y\Vert^2 & \text{Cauchy-Schwarz} \\
    &= (\Vert x\Vert + \Vert y\Vert)^2 \\
    \Vert x + y \Vert &\le \Vert x\Vert + \Vert y\Vert.
  \end{align*}
\end{proof}

\pagebreak

\begin{exercise}
  Let $X$ denote the subset of $\mathbb{R}^\omega$ consisting of all sequences $(x_1,x_2,\dots)$ such that $\sum x_i^2$ converges.
  \begin{enumerate}[(a)]
    \item Show that if $x,y\in X$, then $\sum |x_iy_i|$ converges.
    \item Show that $X$ is closed under addition and scalar multiplication.
    \item Show that
    $$d(x,y) = \left[\sum_{i=1}^{\infty} (x_i - y_i)^2\right]^{1/2}$$
    is a well-defined metric on $X$.
  \end{enumerate}
\end{exercise}
\hrule
\begin{proof}[Solution]
  (a) By the Cauchy-Schwarz inequality and some basic limits, we have $$\sum |x_iy_i| \le \sqrt{\sum x_i^2 \sum y_i^2}.$$
  
  (b) We have
  \begin{align*}
    \sum (x_i + y_i)^2 &= \sum (x_i^2 + 2x_iy_i + y_i^2) \\
    \sum (cx_i)^2 &= c\sum x_i^2,
  \end{align*}
  so sums and scalar products converge.

  (c) $d$ being well defined follows from (b). For the triangle inequality, we can follow a similar proof to Exercise 9.
\end{proof}

\pagebreak

\begin{exercise}
  Show that if $d$ is a metric on $X$, then
  $$d'(x,y) = \frac{d(x,y)}{1+d(x,y)}$$
  is a bounded metric which gives the topology of $X$.
\end{exercise}
\hrule
\begin{proof}[Solution]
  \begin{itemize}
    \item $d'$ is bounded above by $1$.
    \item Note that we have $d'(x,y) \le d(x,y)$, thus $B_d(x,\epsilon)\subseteq B_{d'}(x,\epsilon)$.
    \item Fix $\epsilon > 0$. If $a\in B_{d'}(x,\frac{\epsilon}{1+\epsilon})$, then
      $$\frac{d(x,a)}{1+d(x,a)} < \frac{\epsilon}{1+\epsilon},$$
      so $d(x,a) < \epsilon$ since the function $f(x) = \frac{x}{1+x}$ is strictly increasing. Therefore, $B_{d'}(x,\frac{\epsilon}{1+\epsilon})\subseteq B_d(x,\epsilon)$.
  \end{itemize}
  Now we prove the triangle inequality for $d'$. We have
  \begin{align*}
    f(x+y) - f(x) &= \frac{y}{(1+x)(1+x+y)}\le \frac{y}{1+y} = f(y) \\
    d'(x,z) &\le f(d(x,y) + d(y,z)) \le d'(x,y) + d'(y,z).
  \end{align*}

  This process works whenever $f$ is increasing and we can show that $f(x+y)\le f(x) + f(y)$ for all $x,y\ge 0$.
\end{proof}

\pagebreak

\end{document}