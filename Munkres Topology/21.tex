\documentclass{article}
\usepackage{amsmath, amsfonts, amssymb, amsthm, enumerate}
\theoremstyle{definition}
\newtheorem{exercise}{Exercise}[section]

\begin{document}
\addtocounter{section}{20}
\section{Metric Spaces (Continued)}

\begin{exercise}
  Let $A\subseteq X$. If $d$ is a metric for the topology of $X$, show that $d|A\times A$ is a metric for the subspace topology of $A$.
\end{exercise}
\hrule
\begin{proof}[Solution]
  Let $d' = d|A\times A$. We have
  $$B_{d'}(x,\epsilon) = \{a\in A\mid d(a,x) < \epsilon\} = A\cap B_d(x,\epsilon).$$
\end{proof}

\pagebreak

\begin{exercise}
  Let $X$ and $Y$ be metric spaces with metrics $d_X$ and $d_Y$, respectively. Let $f:X\to Y$ have the property that for every pair $x_1,x_2\in X$,
  $$d_Y(f(x_1),f(x_2)) = d_X(x_1,x_2).$$
  Show that $f$ is an imbedding. It is called an isometric imbedding of $X$ in $Y$.
\end{exercise}
\hrule
\begin{proof}[Solution]
  That $f$ is continuous follows directly from the distance-preserving property. Also, $f$ is injective, since if $f(x_1) = f(x_2)$, then $$d_Y(f(x_1),f(x_2)) = d_X(x_1,x_2) = 0,$$
  thus $x_1 = x_2$. The inverse function $f^{-1}$ is continuous because it's also distance-preserving. Therefore, $f$ is an embedding.
\end{proof}

\pagebreak

\begin{exercise}
  Let $X_n$ be a metric space with metric $d_n$, for $n\in\mathbb{Z}_+$.
  \begin{enumerate}[(a)]
    \item Show that
    $$\rho(x,y) = \max\{d_1(x_1,y_1),\dots,d_n(x_n,y_n)\}$$
    is a metric for the product space $X_1\times \dots \times X_n$.
    \item Let $\bar{d_i} = \min\{d_i,1\}$. Show that
    $$D(x,y) = \sup\{\bar{d_i}(x_i,y_i)/i\}$$
    is a metric for the product space $\prod X_i$.
  \end{enumerate}
\end{exercise}
\hrule
\begin{proof}[Solution]
  (a) Open balls are open in the product topology. Now suppose $U = \prod B_{d_i}(x_i, \epsilon_i)$ is a basis element of the product topology and $a\in U$. Then we have $a_i \in B_{d_i}(x_i, \epsilon)$, so we can find some $\delta_i$ such that
  $$B_{d_i}(a_i, \delta_i) \subseteq B_{d_i}(x_i, \epsilon).$$
  Let $\delta = \min\{\delta_i\}$, and we have $B_\rho(a, \delta)\subseteq U$.

  (b) It's easy to check that any open set in the product topology contains an open ball with respect to the metric $D$. Now let $B_D(x,\epsilon)$ be the neighborhood of $x$ with radius $\epsilon$. Define the open set $U$ by the equation
  $$U = \prod_{i=1}^{\infty}B_{\bar{d_i}}(x_i,\epsilon i).$$
  But we have $B_{\bar{d_i}}(x_i,\epsilon i) = \mathbb{R}$ for all $i > 1/\epsilon$, thus $U$ is open in the product topology. By our chosen definition, we have $x\in U\subseteq B_D(x,\epsilon)$, so the metric $D$ does indeed induce the product topology.
\end{proof}

\pagebreak

\begin{exercise}
  Show that $\mathbb{R}_\ell$ and the ordered square satisfy the first countability axiom.
\end{exercise}
\hrule
\begin{proof}[Solution]
  Every neighborhood of $x$ in $\mathbb{R}_n$ contains some set of the form $[x, x+1/n)$.

  If $x\times y$ is a point on the ordered square and $y\ne 1$, then every neighborhood of $x\times y$ in the ordered square topology contains some set $$\{x\}\times(y-1/n, y+1/n).$$

  If $y = 1$, then every neighborhood of $x\times 1$ contains some interval $$((x\times y-1/n), (x+1/n\times y)).$$
\end{proof}

\pagebreak

\begin{exercise}
  Let $x_n \to x$ and $y_n \to y$ in the space $\mathbb{R}$. Then
  \begin{align*}
    x_n + y_n &\to x + y \\
    x_n - y_n &\to x - y \\
    x_ny_n &\to xy \\
    x_n / y_n &\to x / y & \text{if $y\ne 0$ and $y_n\ne 0$.}
  \end{align*}
\end{exercise}
\hrule
\begin{proof}[Solution]
  Let $\epsilon > 0$. For addition, we have
  $$|(x_n + y_n) - (x + y)| \le |x_n - x| + |y_n - y| < \epsilon$$
  whenever $|x_n - x| < \epsilon/2$ and $|y_n - y| < \epsilon$, which is true for almost all $n$. Subtraction is very similar.

  For multiplication, let $M$ be greater than all of $|y|,|x_1|,|x_2|,\dots$. Then we have
  \begin{align*}
    |x_ny_n - xy| &= |x_ny_n - x_ny + x_ny - xy| \\
    &\le |x_n||y_n - y| + |y||x_n - x| \\
    &\le M(|x_n - x| + |y_n - y|) < \epsilon
  \end{align*}
  when $|x_n - x| < \epsilon/2M$ and $|y_n - y| < \epsilon/2M$, which is true for almost all $n$.

  For reciprocals, choose $\delta > 0$ such that $|x_n| > \delta$ for all $n$, and also $|x| > \delta$. Then we have
  $$\left|\frac{1}{x_n} - \frac{1}{x}\right| = \frac{|x_n - x|}{|x_n||x|} \le \frac{|x_n - x|}{\delta^2} < \epsilon$$
  whenever $|x_n - x| < \delta^2 \epsilon$. Then convergence of division follows from this combined with multiplication.
\end{proof}

\pagebreak

\begin{exercise}
  Define $f_n:[0,1]\to\mathbb{R}$ by the equation $f_n(x) = x^n$. Show that the sequence $(f_n(x))$ converges for each $x\in[0,1]$, but that the sequence $(f_n)$ doesn't converge uniformly.
\end{exercise}
\hrule
\begin{proof}[Solution]
  The sequences $(f_n(x))$ all converge by the monotone convergence theorem. Also, we have
  $$\lim x^n = \lim x^{n+1} = x\lim x^n,$$
  thus $(f_n(x))\to 0$ except when $x = 1$, where $(f_n(1))\to 1$. This limit function is clearly not continuous, so the sequence $(f_n)$ couldn't have converged uniformly.
\end{proof}

\pagebreak

\begin{exercise}
  Let $X$ be a set and let $f_n:X\to\mathbb{R}$ be a sequence of functions. Let $\bar{\rho}$ be the uniform metric on $\mathbb{R}^X$. Show that the sequence $(f_n)$ converges uniformly to the function $f:X\to\mathbb{R}$ if and only if $(f_n)$ converges to $f$ as elements of the metric space $(\mathbb{R}^X,\bar{\rho})$.
\end{exercise}
\hrule
\begin{proof}[Solution]
  $(f_n)$ converging to $f$ in the uniform metric is equivalent to the condition that for all $\epsilon > 0$ we have
  $$\sup_{x\in X}|f_n(x) - f(x)| \le \epsilon$$
  for almost all $n$. This is equivalent to $|f_n(x) - f(x)| \le \epsilon$ for all $x$ and almost all $n$, which is the same as $(f_n)$ uniformly converging to $f$. We can use nonstrict inequalities because $0 < \epsilon/2 < \epsilon$ for all $\epsilon > 0$.
\end{proof}

\pagebreak

\begin{exercise}
  Let $X$ be a topological space and let $Y$ be a metric space. Let $f_n:X\to Y$ be a sequence of continuous functions. Let $(x_n)$ be a sequence of points of $X$ converging to $x$. Show that if the sequence $(f_n)$ converges uniformly to $f$, then $(f_n(x_n))$ converges to $f(x)$.
\end{exercise}
\hrule
\begin{proof}[Solution]
  Let $\epsilon > 0$. By uniform convergence, for almost all $n$, we have \\ $d(f_n(a), f(a)) < \epsilon/2$ for all $a$. 
  Also, since the limit $f$ is continuous, we have $d(f(x_n), f(x)) < \epsilon / 2$ for almost all $n$. Combining these,
  $$d(f_n(x_n), f(x)| \le d(f_n(x_n), f(x_n)) + d(f(x_n), f(x)) < \epsilon$$
  for almost all $n$, thus $(f_n(x_n))\to f(x)$.
\end{proof}

\pagebreak

\begin{exercise}
  Let $f_n:\mathbb{R}\to\mathbb{R}$ be the function
  $$f_n(x) = \frac{1}{n^3[x-(1/n)]^2 + 1}.$$
  Let $f$ be the zero function.
  \begin{enumerate}[(a)]
    \item Show that $f_n(x)\to f(x)$ for each $x\in\mathbb{R}$.
    \item Show that $(f_n)$ does not converge uniformly to $f$.
  \end{enumerate}
\end{exercise}
\hrule
\begin{proof}[Solution]
  (a) We have $f_n(x)\to 0$ by properties of the convergence of rational functions.

  (b) For all $n$ we have $f_n(1/n) = 1$, thus $(f_n)$ doesn't converge uniformly to $f$.
\end{proof}

\pagebreak

\begin{exercise}
  Using the closed set formulation of continuity, show that the following sets are closed in $\mathbb{R}^2$.
  \begin{align*}
    A &= \{x\times y\mid xy = 1\} \\
    S^1 &= \{x\times y\mid x^2 + y^2 = 1\} \\
    B^2 &= \{x\times y\mid x^2 + y^2 \le 1\}.
  \end{align*}
\end{exercise}
\hrule
\begin{proof}[Solution]
  We will assume arithmetic operators are continuous, so the functions $f(x,y) = xy$ and $g(x,y) = x^2 + y^2$ are both continuous. Then we have
  \begin{align*}
    A &= f^{-1}(\{1\}) \\
    S^1 &= g^{-1}(\{1\}) \\
    B^2 &= g^{-1}((-\infty,1]).
  \end{align*}
\end{proof}

\pagebreak

\end{document}