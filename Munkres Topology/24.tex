\documentclass{article}
\usepackage{amsmath, amsfonts, amssymb, amsthm, enumerate}
\theoremstyle{definition}
\newtheorem{exercise}{Exercise}[section]

\begin{document}
\addtocounter{section}{23}
\section{Connected Subspaces of the Real Line}

\begin{exercise}
  \begin{enumerate}[(a)]
    \item Show that no two of the spaces $(0,1),(0,1],[0,1]$ are homeomorphic.
    \item Suppose that there exists imbeddings $f:X\to Y$ and $g:Y\to X$. Show by means of an example that $X$ and $Y$ need not be homeomorphic.
    \item Show $\mathbb{R}^n$ and $\mathbb{R}$ are not homeomorphic if $n > 1$.
  \end{enumerate}
\end{exercise}
\hrule
\begin{proof}[Solution]
  (a) Following the hint, consider what happens when you remove a point from each space.
  \begin{itemize}
    \item $(0,1)\setminus\{x\}$ is always disconnected.
    \item $(0,1]\setminus\{x\}$ is connected if and only if $x = 1$.
    \item $[0,1]\setminus\{x\}$ is connected if and only if $x = 0$ or $x = 1$.
  \end{itemize}
  The number of points which yield a connected space once removed is invariant under homeomorphic spaces, thus all three intervals are nonhomeomorphic.

  (b) The example they are heavily implying is that any two of the intervals in (a) can be imbedded in one another.

  (c) As was shown in the examples, removing a single point from $\mathbb{R}^n$ yields a connected space for $n > 1$. This is not true for $n = 1$.
\end{proof}

\pagebreak

\begin{exercise}
  Let $f:S^1\to\mathbb{R}$ be continuous. Show that there exists a point $x$ of $S^1$ such that $f(x) = f(-x)$.
\end{exercise}
\hrule
\begin{proof}[Solution]
  Fix any $a$ such that $f(a) \le f(-a)$. Then we have $f(a) - f(-a) \le 0$ and $f(-a) - f(a) \ge 0$. Since $S^1$ is connected, by the intermediate value theorem there is some $x$ such that $f(x) - f(-x) = 0$, or $f(x) = f(-x)$.
\end{proof}

\pagebreak

\begin{exercise}
  Let $f:X\to X$ be continuous. Show that if $X = [0,1]$, there is a point $x$ such that $f(x) = x$. The point $x$ is called a fixed point of $f$. What happens if $X$ is $[0,1)$ or $(0,1)$?
\end{exercise}
\hrule
\begin{proof}[Solution]
  We have $g(x) = x - f(x)$ is continuous, and $g(0) \le 0$ and $g(1) \ge 0$, thus there is some point $x\in[0,1]$ such that $g(x) = 0$, which is the same as $f(x) = x$.

  If $X = [0,1)$, the function $f(x) = \frac{x+1}{2}$ has no fixed point (the fixed point would be $1$). The same applies with $X = (0,1)$.
\end{proof}

\pagebreak

\begin{exercise}
  Let $X$ be an ordered set in the order topology. Show that if $X$ is connected, then $X$ is a linear continuum.
\end{exercise}
\hrule
\begin{proof}[Solution]
  If there were points $x < y$ with no point between them, then the open intervals $(-\infty, y)$ and $(x,\infty)$ form a separation of $X$.

  Suppose that some bounded set $S$ has no least upper bound, then let $B$ be the nonempty set of upper bounds.
  If $x\in B$, there is some $y < x$ such that $y\in B$, so the ray $(y,\infty)$ is a neighborhood of $x$ contained in $B$. This shows $B$ is open.

  If $x\in X\setminus B$, then $x$ is not an upper bound of $S$, thus there is some $s\in S$ greater than $x$. We have $s\in X\setminus B$ as well, for if $s$ was an upper bound, it would be the least upper bound. Therefore, $x\in (-\infty, s)\subseteq X\setminus B$, so $X\setminus B$ is also open.

  Now $B$ and $X\setminus B$ form a separation of $X$. Therefore, if $X$ is connected, it must be a linear continuum.
\end{proof}

\pagebreak

\begin{exercise}
  Consider the following sets in the dictionary order. Which are linear continua?
  \begin{enumerate}[(a)]
    \item $\mathbb{Z}_+\times [0,1)$
    \item $[0,1)\times\mathbb{Z}_+$
    \item $[0,1)\times[0,1]$
    \item $[0,1]\times[0,1)$
  \end{enumerate}
\end{exercise}
\hrule
\begin{proof}[Solution]
  (a) This is a linear continuum by Exercise 6 because $\mathbb{Z}_+$ is well ordered.

  (b) There are no points between $0\times 1$ and $0\times 2$, so this is not a linear continuum.

  (c) Clearly every pair of points has a point between them. Now let $S$ be a nonempty bounded set. Let $x$ be the least upper bound of $\pi_1(S)$. If $x\in\pi_1(S)$, then let
  $$y = \sup\pi_2(\pi_1^{-1}\{x\}\cap S).$$
  Then $(x,y)$ is the least upper bound of $S$.

  If on the other hand $x\notin\pi_1(S)$, then the point $(x,0)$ is the least upper bound of $S$.

  (d) The set $\{0\}\times [0,1)$ has no least upper bound.
\end{proof}

\pagebreak

\begin{exercise}
  If $X$ is a well ordered set, show that $X\times [0,1)$ is a linear continuum.
\end{exercise}
\hrule
\begin{proof}[Solution]
  Clearly there is a point between any two distinct points. Now let $S$ be a nonempty bounded set. Because $X$ is well ordered, the point $x = \sup\pi_1(S)$ exists.

  If $x\notin \pi_1(S)$, then we have $\sup S = (x\times 0)$.

  If $x\in\pi_1(S)$, then let $A = \pi_2(\pi_1^{-1}\{x\}\cap S)$ be the subset of points $a\in [0,1)$ such that $(x\times a)\in S$.
  \begin{itemize}
    \item If $\sup A < 1$, then $\sup S = (x\times a)$.
    \item If $\sup A = 1$, then $\sup S = (x'\times 0)$, where $x'$ is the immediate successor of $x$. If no such successor existed, then $S$ would not be bounded.
  \end{itemize}
\end{proof}

\pagebreak

\begin{exercise}
  \begin{enumerate}[(a)]
    \item Let $X$ and $Y$ be ordered sets in the order topology. Show that if $f:X\to Y$ is order preserving and surjective, then $f$ is a homeomorphism.
    \item Let $X = Y = \bar{\mathbb{R}}_+$. Given a positive integer $n$, show that the function $f(x) = x^n$ is order preserving and surjective. Conclude that it's inverse, the $n^{th}$ root function, is continuous.
    \item Let $X$ be the subspace $(-\infty, -1)\cup[0,\infty)$ of $\mathbb{R}$. Show that the function $f:X\to\mathbb{R}$ defined by setting $f(x) = x+1$ if $x < -1$, and $f(x) = x$ if $x\ge 0$, is order preserving and surjective. Is $f$ a homeomorphism?
  \end{enumerate}
\end{exercise}
\hrule
\begin{proof}[Solution]
  (a) $f$ is bijective by the order preserving property. We have $f((x,y)) = (f(x),f(y))$, showing that $f^{-1}$ is continuous. A similar argument shows that $f$ is continuous, and thus $f$ is a homeomorphism.

  (b) A simple inductive argument shows that $f$ is order preserving. Since $f(0) = 0$ and $f(x) > x$ for all $x > 1$, the intermediate value theorem shows that $f$ is surjective.

  (c) $f$ is order preserving and surjective by casework, or intuition. However, $f$ is not a homeomorphism. This is because $X$ doesn't have the order topology as a subspace of $\mathbb{R}$.
\end{proof}

\pagebreak

\begin{exercise}
  \begin{enumerate}[(a)]
    \item Is the product of path-connected spaces necessarily path connected?
    \item If $A\subseteq X$ and $A$ is path connected, is $\bar{A}$ necessarily path connected?
    \item If $f:X\to Y$ is continuous and $X$ is path connected, is $f(X)$ necessarily path connected?
    \item If $A_\alpha$ is a collection of path connected subspaces of $X$ and if $\bigcap A_\alpha\ne\emptyset$, is $\bigcup A_\alpha$ necessarily path connected?
  \end{enumerate}
\end{exercise}
\hrule
\begin{proof}[Solution]
  (a) Let $a = (a_\alpha)$ and $b = (b_\alpha)$ be two points in $X = \prod X_\alpha$. For each $\alpha$, let $p_\alpha:[0,1]\to X_\alpha$ be a path from $a_\alpha$ to $b_\alpha$. Define
  $$p(t) = (p_\alpha(t))_{\alpha\in J}.$$
  If $U = \pi_\alpha^{-1}(U_\alpha)$ is a subbasis element of the product space, then
  $$p^{-1}(U) = (\pi_\alpha\times p)^{-1}(U) = p_\alpha^{-1}(U),$$
  which is open. This shows that the path $p$ is continuous, thus $X$ is path connected.

  (b) No, the topologist's sine curve is a counterexample.

  (c) Let $f(a),f(b)\in f(X)$; let $g:[0,1]\to X$ be a path from $a$ to $b$. Then $f\circ g$ is a path from $f(a)$ to $f(b)$. Therefore, $f(X)$ is path connected.

  (d) Let $a$ be any point in $\bigcap A_\alpha$; choose $x\in A_\alpha$ and $y\in A_\beta$. Then we can define paths
  $$f:[0,1]\to A_\alpha,\qquad g:[1,2]\to A_\beta$$
  such that $f(0) = x$, $f(1) = g(1) = a$, and $g(2) = y$. Then the path $h$ formed by combining the domains of $f$ and $g$ is continuous by the pasting lemma. This shows that $\bigcup A_\alpha$ is path connected.
\end{proof}

\pagebreak

\begin{exercise}
  Assume that $\mathbb{R}$ is uncountable. Show that if $A\subseteq\mathbb{R}^2$ is countable, then $\mathbb{R}^2 - A$ is path connected.
\end{exercise}
\hrule
\begin{proof}[Solution]
  Choose $x,y\in\mathbb{R}^2 - A$. Following the hint, there are uncountably many lines through $x$ which don't intersect $A$; choose one of them. There are also uncountably many lines through $y$ which don't intersect $A$; choose one of them which intersects the first line. This forms our path.
\end{proof}

\pagebreak

\begin{exercise}
  Show that if $U$ is an open connected subspace of $\mathbb{R}^2$, then $U$ is path connected.
\end{exercise}
\hrule
\begin{proof}[Solution]
  There is a general way to determine whether a property $P(x)$ holds for all points in a connected set $C$. It has a similar feel to induction.
  \begin{itemize}
    \item Find an initial point $x_0\in C$ which satisfies $P$.
    \item Show that $\{x\in C\mid P(x)\}$ is both open and closed in $C$.
  \end{itemize}
  Since $C$ is connected, this would show that every $x$ satisfies the property.

  Let $U$ be an open and connected subspace of $\mathbb{R}^2$. We can assume that $U$ is nonempty, since the empty set is path connected. Let $x_0\in U$. We will show that the set $A$ containing all points which can be connected to $x_0$ with a path is open and closed in $U$, and therefore $A = U$.

  Let $x\in A$. Since $U$ is open, we can choose some $B(x,\epsilon)$ contained in $U$, which is path connected. Therefore, $B(x,\epsilon)$ is contained in $A$, and so $A$ is open.

  Now let $x\notin A$. Again, we can choose a path connected ball $B(x,\epsilon)$ contained in $U - A$. If the open ball intersected $A$, this would mean $x\in A$, which is a contradiction. This shows that $A$ is closed.

  Since $C$ is connected and $A\subseteq C$ is nonempty, open, and closed, we have $A = C$. This shows that $C$ is path connected.
\end{proof}

\pagebreak

\begin{exercise}
  If $A$ is a connected subspace of $X$, does it follow that $\mathrm{Int}(A)$ and $\mathrm{Bd}(A)$ are connected? Does the converse hold?
\end{exercise}
\hrule
\begin{proof}[Solution]
  If $A$ is connected, neither the interior nor the boundary need to be connected. For an informal example, let $A\subseteq\mathbb{R}^2$ be two balls connected by a line, where the interior removes the connecting line. Also, $\mathrm{Bd}((0,1)) = \{0,1\}$ is clearly not connected.

  The converse also doesn't hold! For example, let $A = \mathbb{Q}$, where $\mathrm{Int}(A) = \emptyset$ and $\mathrm{Bd}(A) = \mathbb{R}$ are both connected.
\end{proof}

\pagebreak

\end{document}