\documentclass{article}
\usepackage{amsmath, amsfonts, amssymb, amsthm, enumerate}
\theoremstyle{definition}
\newtheorem{exercise}{Exercise}[section]

\begin{document}
\addtocounter{section}{12}
\section{Topological Spaces / Bases}

\begin{exercise}
  Let $X$ be a topological space; let $A$ be a subset of $X$. Suppose that for each $x\in A$ there is an open set $U$ containing $x$ such that $U\subseteq A$. Show that $A$ is open in $X$.
\end{exercise}
\begin{proof}
  For each $x$, let $U_x$ be an open set containing $x$ such that $U_x\subseteq A$.
  (Note that the axiom of choice isn't required, as we can let $U_x$ be the largest such open set.) Then I claim that
  $$A = \bigcup_{x\in A} U_x.$$
  If $x\in A$, then $x\in U_x$ by definition. The reverse inclusion holds since we have $U_x\subseteq A$ for each $x$. Now $A$ is the union of open sets, and so $A$ is open.
\end{proof}

\begin{exercise}
  Consider the nine topologies on the set $X = \{a, b, c\}$ indicated in Example 1
 of Section 12. Compare them, that is, for each pair of topologies, determine whether
 they are comparable, and if so, which is the finer.
\end{exercise}
\begin{proof}[Solution]
  Skipped, as this exercise is trivial.
\end{proof}

\begin{exercise}
  Show that $\mathcal{T}_c$, the \textit{countable complement topology}, is a topology on a given set $X$.
  Is the collection
  $$\mathcal{T}_{\infty} = \{U\mid \text{$X-U$ is infinite or empty}\}$$
  a topology?
\end{exercise}
\begin{proof}[Solution]
  Clearly $\emptyset$ and $X$ are open in $\mathcal{T}_c$. If $(U_\alpha)$ is a collection of open sets, then
  $$X - \bigcup_{\alpha}U_\alpha = \bigcap_{\alpha}(X - U_\alpha),$$
  which is a subset of some countable $X - U_\alpha$, and is therefore countable. The exception is when $X - U_\alpha = X$ for every $\alpha$, in which case the intersection is $X$, which is allowed by our definition.

  Let $(U_j)$ be a finite collection of $n$ open sets. Then
  $$X - \bigcap_{j=1}^n U_j = \bigcup_{j=1}^n (X - U_j).$$
  A finite union of countable sets is countable, and if $X - U_j = X$ for some $j$, then the union is $X$.

  Since arbitrary unions and finite intersections of open sets are open, $\mathcal{T}_c$ is a topology.

  Consider $\mathcal{T}_\infty$ on the set $\mathbb{N}$. Then $\{2,4,6,\dots\}$ and $\{3,5,7,\dots\}$ are both open, but their union has complement $\{1\}$ and is not open, so $\mathcal{T}_\infty$ is not a topology.
\end{proof}

\begin{exercise}
  \begin{enumerate}[(a)]
    \item If $\{\mathcal{T}_\alpha\}$ is a collection of topologies on $X$, then $\bigcap T_\alpha$ is a topology. Is $\bigcup T_\alpha$ a topology?
    \item Let $\{\mathcal{T}_\alpha\}$ be a family of topologies on $X$. Show that there is a smallest topology which contains every $\mathcal{T}_\alpha$ (their least upper bound).
    \item If $X = \{a,b,c\}$, let
    $$\mathcal{T}_1 = \{\emptyset, X, \{a\}, \{a,b\}\} \qquad\text{and}\qquad \mathcal{T}_2 = \{\emptyset, X, \{a\}, \{b,c\}\}.$$
    Find the intersection and least upper bound topologies
  \end{enumerate}
\end{exercise}
\begin{proof}[Solution]
  (a) Every $\mathcal{T}_\alpha$ has both $\emptyset$ and $X$ open, so the intersection has these. Now consider a family of open sets $\{U_\beta\}$ in $\bigcap T_\alpha$. Then they are in each $\mathcal{T}_\alpha$, thus their union is also in the intersection topology. A similar argument applies to finite intersection.

  Let $X$ have at least 3 elements and let $\mathcal{T}_1 = \{\emptyset, \{a\}, X\}$ and $\mathcal{T}_2 = \{\emptyset, \{b\}, X\}$. Then $\mathcal{T}_1\cup\mathcal{T}_2$ is not a topology.

  (b) Note that the discrete topology contains \textit{every} topology, so such topologies exist. Then we can let $\mathcal{T}_{\sup}$ be the intersection of all these by part (a).

  (c) The intersection is $\{\emptyset, X, \{A\}\}$, and the least upper bound is $$\{\emptyset, X, \{a\}, \{b\}, \{a,b\}, \{b,c\}\}$$
\end{proof}

\begin{exercise}
  Show that if $\mathcal{A}$ is a basis for a topology on $X$, then the topology generated by $\mathcal{A}$ is the intersection of the topologies containing $\mathcal{A}$. Prove the same if $\mathcal{A}$ is a subbasis
\end{exercise}
\begin{proof}[Solution]
  Suppose $\mathcal{A}$ is a basis of $\mathcal{T}$. Let $\mathcal{T}'$ be any topology containing all of $\mathcal{A}$. Then $\mathcal{T}'$ contains every possible union of $\mathcal{A}$, which is just $\mathcal{T}$.

  Now suppose $\mathcal{A}$ is instead a subbasis generating $\mathcal{T}$. Again let $\mathcal{T}'$ be a topology containing all of $\mathcal{A}$. Then $\mathcal{T}'$ contains every combination of unions and finite intersections of sets in $\mathcal{A}$, so $\mathcal{T}\subseteq\mathcal{T}'$.
\end{proof}

\begin{exercise}
  Show that the topologies of $\mathcal{R}_\ell$ and $\mathbb{R}_K$ are incomparable.
\end{exercise}
\begin{proof}[Solution]
  No basis element of $\mathbb{R}_K$ lies inside $[-1,1)$ and contains $-1$.

  No basis element of $\mathbb{R}_\ell$ lies inside $(-1,1) - K$ and contains $0$ by the\\ Archimedean principle.
\end{proof}

\begin{exercise}
  Consider the following topologies on $\mathbb{R}$:
  \begin{itemize}
    \item[] $\mathcal{T}_1 = $ the standard topology
    \item[] $\mathcal{T}_2 = \mathbb{R}_K$
    \item[] $\mathcal{T}_3 = $ the finite complement topology
    \item[] $\mathcal{T}_4 = $ the upper limit topology
    \item[] $\mathcal{T}_5 = $ the topology having all sets $(-\infty, a)$ as a basis.
  \end{itemize}
  Now compare them all
\end{exercise}
\begin{proof}[Solution]
  ($\mathcal{T}_3\subset\mathcal{T}_1\subset\mathcal{T}_2,\mathcal{T}_4$)
  We've seen that $\mathcal{T}_1\subseteq\mathcal{T}_2,\mathcal{T}_4$ in the textbook. Let $U$ be open in $\mathcal{T}_3$. If $U$ is empty, we're done. Otherwise, choose $x\in U$. Since there are only finitely many gaps in $U$, we can find the minimum distance $\varepsilon$ from $x$ to a gap. Then $(x - \varepsilon, x + \varepsilon)$ is an open subset of $U$ in $\mathcal{T}_1$ and contains $x$. Since $x$ was arbitrary, $U$ is open in $\mathcal{T}_1$.

  ($\mathcal{T}_3\perp\mathcal{T}_5$) We have $\mathbb{R}-\{0\}$ is open in $\mathcal{T}_3$ but not $\mathcal{T}_5$. Also, $(-infty, 0)$ is open in $\mathcal{T}_5$, but not $\mathcal{T}_3$.

  ($\mathcal{T}_5\subset \mathcal{T}_1$) Every basis element $(-\infty,a)$ of $\mathcal{T}_5$ is also open in $\mathcal{T}_1$. On the other hand, $(0,1)$ is open in $\mathcal{T}_1$, but not $\mathcal{T}_5$.

  ($\mathcal{T}_2\perp\mathcal{T}_4$) This argument is similar to Exercise 6.

  These comparisons are sufficient. 
\end{proof}

\begin{exercise}
  \begin{enumerate}[(a)]
    \item Show that the countable collection
    $$\mathcal{B} = \{(a,b)\mid a < b\text{ and }a,b\in\mathbb{Q}\}$$
    is a basis which generates the standard topology on $\mathbb{R}$.
    \item Show that the collection
    $$\mathcal{B} = \{[a,b)\mid a < b\text{ and }a,b\in\mathbb{Q}\}$$
    doesn't generate the lower limit topology on $\mathbb{R}$.
  \end{enumerate}
\end{exercise}
\begin{proof}[Solution]
  (a) Clearly every element of $\mathcal{B}$ is open in the standard topology. Now let $(x,y)$ be a basis set of the standard topology, and choose $z$ in the interval. By density of $\mathbb{Q}$, there are rational numbers $a$ and $b$ such that $x < a < z < b < y$, so that $z$ is in the basis set $(a,b)\in\mathcal{B}$.

  (b) If $a$ is irrational, then $[a, \infty)$ is open in $\mathbb{R}_\ell$, but no element of $\mathcal{B}$ contained inside $[a,\infty)$ has $a$ in it. In fact, $\mathbb{R}_\ell$ has no countable basis.
\end{proof}


\end{document}