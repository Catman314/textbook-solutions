\documentclass{article}
\usepackage{amsmath, amsfonts, amssymb, amsthm, enumerate}
\theoremstyle{definition}
\newtheorem{exercise}{Exercise}

\begin{document}
\section*{Cyclic Groups}

\setcounter{exercise}{15}

\begin{exercise}
  Assume $|x| = n$ and $|y| = m$. Suppose that $x$ and $y$ commute. Show that $|xy|$ divides $\mathrm{lcm}(n,m)$. Need this be true if $x$ and $y$ don't commute? Give an example of commuting elements $x$ and $y$ such that $|xy|$ is a proper divisor of $\mathrm{lcm}(n,m)$.
\end{exercise}
\hrule
\begin{proof}[Solution]
  Letting $\ell$ be the least common multiple of $|x|$ and $|y|$, we have $(xy)^{\ell} = x^\ell y^\ell = 1$, therefore $|xy|$ divides $\ell$.

  In $D_8$, the elements $s$ and $sr$ have order $2$, but $s\cdot sr = r$ has order $4$, which doesn't divide $2$.

  If $y = x^{-1}$, then $xy$ has order 1, which is a proper divisor of $|x|$ as long as $x\ne 1$.
\end{proof}

\pagebreak

\setcounter{exercise}{17}

\begin{exercise}
  If $h\in H$ with $h^n = 1$, then there is a unique homomorphism $Z_n = \langle x\rangle\to H$ such that $x\mapsto h$,
\end{exercise}
\hrule
\begin{proof}[Solution]
  Uniqueness follows by induction, showing that $x^k\mapsto h^k$ for all integers $k$. Now we need to show that this is well defined.

  Suppose that $x^j = x^k$. Then we have $x^{j-k} = 1$, so $n\mid j-k$. But this means that $h^{j-k} = 1$, and so $h^j = h^k$.

  Exercise 19 is even simpler.
\end{proof}

\pagebreak

\setcounter{exercise}{19}

\begin{exercise}
  Let $p$ be prime and $n$ a positive integer. If $x\in G$ such that $x^{p^n} = 1$, show that the order of $x$ is $p^m$ for some integer $m\le n$.
\end{exercise}
\hrule
\begin{proof}[Solution]
  The order of $x$ divides $p^n$, and every divisor of $p^n$ is of the form $p^m$ for some $m\le n$.
\end{proof}

\pagebreak

\begin{exercise}
  Let $p$ be an odd prime and let $n$ be a positive integer. Show that $1+p$ has order $p^{n-1}$ in the multiplicative group $(\mathbb{Z}/p^n\mathbb{Z})^\times$.
\end{exercise}
\hrule
\begin{proof}[Solution]
  By the binomial theorem, we have $$(1+p)^{p^{n-1}} = \sum_{j=0}^{p^{n-1}}\binom{p^{n-1}}{j}p^j.$$

  We will now look more closely at the factors of $p$ in each summand. We have
  \begin{align*}
    \nu_p\left(\binom{p^{n}}{j}p^j\right) &= \nu_p(p^{n}!) - \nu_p(j!) - \nu_p((p^{n}-j)!) + j \\
    &= \frac{p^n-1}{p-1} - \sum_{i = 1}^n\lfloor{j/p^i}\rfloor - \sum_{i=1}^n\lfloor(p^n-j)/p^i\rfloor + j \\
    &= - \sum_{i = 1}^n\lfloor{j/p^i}\rfloor - \sum_{i=1}^n\lfloor-j/p^i\rfloor + j \\
    &= \sum_{i=1}^n (\lceil j/p^i\rceil - \lfloor j/p^i\rfloor) + j \\
    &= n + j - \nu_p(j) & j > 0.
  \end{align*}
  Since $n + j - \nu_p(j) \ge n+1$, we can conclude that $p^{n+1}\mid \binom{p^{n}}{j}p^j$ for all $j > 0$. In particular, every term $j>0$ in the binomial sum is reduced to 0 modulo $p^n$, so we get
  $$(1+p)^{p^{n-1}} \equiv 1 \pmod{p^n}.$$

  Also, we have
  $$(1+p)^{p^{n-2}} = \sum_{j=0}^{p^{n-2}}\binom{p^{n-2}}{j}p^j,$$
  and since $n + j - \nu_p(j) \ge n+2$ for all $j\ge 2$ and $p > 2$, we reduce the terms with $j\ge 2$ to 0 modulo $p^n$, leaving us with
  $$(1+p)^{p^{n-2}} \equiv 1 + p^{n-1} \not\equiv 1 \pmod{p^n}$$

  For problem 22 where $p=2$, this second part doesn't work, but we have $n + j - \nu_2(j) \ge n+2$ for all $j\ge 3$. In fact, the expression reduces to $1$ modulo $2^n$ whenever $n \ge 3$.

\end{proof}

\pagebreak

\begin{exercise}
  Let $n\ge 3$. Show that $5$ has order $2^{n-2}$ in the group $(\mathbb{Z}/2^n\mathbb{Z})^\times$.
\end{exercise}
\hrule
\begin{proof}[Solution]
  In Exercise 21, we derived that $\nu_p\left(\binom{p^n}{j}\right) = n - \nu_p(j)$. We have
  $$(1+2^2)^{2^{n-2}} = \sum_{j=0}^{2^{n-2}}\binom{2^{n-2}}{j}2^{2j}.$$
  The $2$-adic valuation of the $j^{th}$ term is $n+2j-\nu_2(j)-2$ for $j>0$, which means we can reduce every term except the first to 0, giving our result.

  On the other hand,
  $$(1+2^2)^{2^{n-3}} = \sum_{j=0}^{2^{n-3}}\binom{2^{n-3}}{j}2^{2j}.$$
  The $2$-adic valuation of the $j^{th}$ term is $n+2j-\nu(j)-3$, so we can reduce every term except the first two. The reduced form is $1+2^{n-1}$, which is not $1$.
\end{proof}

\pagebreak

\begin{exercise}
  Show that $(\mathbb{Z}/2^n\mathbb{Z})^\times$ is not cyclic for $n\ge 3$.
\end{exercise}
\hrule
\begin{proof}[Solution]
  The elements $2^n - 1$ and $2^{n-1} + 1$ are distinct and both have order 2, but there can be only one cyclic subgroup of each order in a finite cyclic group, so $(\mathbb{Z}/2^{n}\mathbb{Z})^\times$ is not cyclic.
\end{proof}

\pagebreak

\begin{exercise}
  Let $G$ be a finite group and $x\in G$.
  \begin{enumerate}[(a)]
    \item Show that if $g\in N_G(\langle x\rangle)$, then $gxg^{-1} = x^a$ for some $a\in \mathbb{Z}$.
    \item Prove the converse.
  \end{enumerate}
\end{exercise}
\hrule
\begin{proof}[Solution]
  (a) If $g\in N_G(\langle x\rangle)$, then $g\langle x\rangle g^{-1} = \langle x\rangle$, so in particular $gxg^{-1} = x^a$ for some integer $a$.
  
  \vspace{0.5em}
  (b) Suppose $gxg^{-1} = x^a$ for some $a$. Then we have $gx^kg^{-1} = (gxg^{-1})^k = x^{ak}$, therefore $g\langle x\rangle g^{-1}\subseteq \langle x\rangle$.

  The function $y \mapsto gyg^{-1}$ is injective, so since $G$ is a finite group we have $|g\langle x\rangle g^{-1}| = |\langle x\rangle|$, so they are the same set. In other words, $g\in N_G(\langle x\rangle)$.
\end{proof}

\pagebreak

\begin{exercise}
  Let $G$ be a finite group of order $n$ and let $k$ be relatively prime with $n$. Show that $x\mapsto x^k$ is surjective.
\end{exercise}
\hrule
\begin{proof}[Solution]
  Let $y\in G$. We will use the fact that $y^n = 1$. Since $n$ and $k$ are relatively prime, we can find integers $a,b$ such that $na + kb = 1$. Then
  $$(y^b)^k = y^{bk-1}y = y^{bk+na-1}y = y,$$
  so $y^b$ is a $k^{th}$ root of $y$.
\end{proof}

\pagebreak

\begin{exercise}
  Let $Z_n$ be a cyclic group of order $n$ and for each integer $a$ let $\sigma_a(x) = x^a$.
  \begin{enumerate}[(a)]
    \item Show that $\sigma_a$ is an automorphism if and only if $a$ and $n$ are relatively prime.
    \item Prove that $\sigma_a = \sigma_b$ if and only if $a\equiv b\pmod m$.
    \item Prove that every automorphism of $Z_n$ is equal to $\sigma_a$ for some $a$.
    \item Prove that $\sigma_a\circ\sigma_b=\sigma_{ab}$. Deduce that $\mathrm{Aut} Z_n \cong (\mathbb{Z}/n\mathbb{Z})^\times$.
  \end{enumerate}
\end{exercise}
\hrule
\begin{proof}[Solution]
  Suppose $y$ is a generator of $Z_n$.

  (a) $\sigma_a$ is an endomorphism since $Z_n$ is abelian. By Exercise 25, if $\gcd(a,n) = 1$, then $\sigma_a$ is surjective, and therefore bijective since $Z_n$ is finite. And if $g = \gcd(a,n) > 1$ and the group is generated by some $y$, then
  $$\sigma_a(y^{n/g}) = y^{an/g} = 1,$$
  so $\sigma_a$ is not injective.

  \vspace{0.5em}
  (b) Suppose $\sigma_a = \sigma_b$. In particular, this means $y^a = y^b$, so $y^{a-b} = 1$, thus $n\mid a-b$. The other direction is easy.

  \vspace{0.5em}
  (c) Let $\phi$ be an automorphism where $\phi(y) = y^a$. Then
  $$\phi(y^b) = \phi(y)^b = (y^a)^b = (y^b)^a = \sigma_a(y^b).$$

  \vspace{0.5em}
  (d) We have
  $$(\sigma_a\circ\sigma_b)(y^c) = ((y^c)^b)^a = (y^c)^{ab} = \sigma_{ab}(y^c).$$
  This means $\bar{a}\mapsto\sigma_a$ is a homomorphism from $(\mathbb{Z}/n\mathbb{Z})^\times\to \mathrm{Aut}Z_n$, and it is an isomorphism by either one of parts (b) or (c).
\end{proof}

\pagebreak

\end{document}